\documentclass[paper=a4,fontsize=12pt]{scrartcl}
\usepackage[automark]{scrpage2}
\clearscrheadfoot                                  % clear head and foot
\chead{\pagemark}
\ihead{Qtel User Documentation}
\ohead{\headmark}
\setheadsepline{1pt}
\usepackage[english]{babel}
\usepackage[T1]{fontenc}                            % char output
\usepackage[utf8]{inputenc}                         % char input
\usepackage{graphicx}
\usepackage{float}
             
\begin{document}                                    % begin of document
\titlehead{
  \vspace*{1\baselineskip}
  \begin{center}
    \large SvxLink Project\\
    \normalsize http://svxlink.org/
  \end{center}
  \vspace{5,1\baselineskip}}
\subject{Qtel}
\title{User Documentation}
\author{Tobias Blomberg\\SM0SVX}
\date{\today}
\maketitle
\thispagestyle{empty}
\newpage
                                     
\pagestyle{scrheadings}
\tableofcontents
\listoffigures
\pagebreak

   
   \noindent{Qtel stands for the ``QT EchoLink'' client.{\kern3.5pt}It is only a EchoLink client application.{\kern2pt}There is no
   ``sysop mode''.{\kern3.5pt}If it is a link you want to run, have a look at the SvxLink server application.{\kern2pt}Qtel is quite
   a simple application, so most things should be self explainatory. There are three windows: the main window, the communications
   dialog and the settings dialog.}

\section{The Qtel Main Window}
   \begin{figure}[H]
      \centering
      \includegraphics[height=80mm]{qtel-mainwindow.pdf}
      \caption[The Qtel Main Window]{Main Window}
   \end{figure}

   The main window consists of four areas.{\kern2pt}The top-left is where to choose which class of stations to view.{\kern2.5pt}All stations
   in a class is shown to the top-right.{\kern3pt}There are four station classes:{\kern2.5pt}Conferences, Links, Repeaters and (private)
   Stations.{\kern2.5pt}There also is a bookmark class which isn't a real class. It's a collection of your favorite stations. Just 
   right-click on a station in any of the other views to add it to the bookmark list. At left-bottom is a message area that shows the
   messages from the EchoLink directory server. At right-bottom is a list of incoming connections.{\kern2pt}To accept an incoming 
   connection, highlight the station and press ``Accept''.

\section{The Qtel Communications Dialog}
   \begin{figure}[H]
      \centering
      \includegraphics[height=80mm]{qtel-comdialog.pdf}
      \caption[The Qtel Communications Dialog]{Communication Dialog}
   \end{figure}

   The communications dialog is used to perform a Qso to anoher station.{\kern2.5pt}To activate it, double-click on a station line or just 
   press the enter key when the station is highlighted.{\kern2.5pt}To connect to the station, press the Connect button. When audio is coming 
   in from the remote station, the green RX indicator will light up.{\kern2.5pt}To transmit, press the PTT button.{\kern3pt}The red TX 
   indicator will light up.{\kern2.5pt}To lock the PTT in transmit, first press and hold the Ctrl-key and then press the PTT
   button.{\kern2.5pt}To disconnect, press the ``Disconnect'' button.{\kern3pt}The VOX is used to automatic activate the PTT when speaking
   into the microphone.{\kern2.5pt}To enable the VOX, check the ``VOX Enabled'' check box. Then set a suitable level for ``Threshold'' so 
   that the VOX does not trigger by mistake on noise. Adjust the ``Delay'' slider so that the hangtime is long enough to allow speech pauses 
   without the PTT getting released. In the big, white area in the middle of the window, info messages from the remote station are 
   shown.{\kern3pt}Chat messages from the remote station is also shown here. In the oneline white area below, chat messages can be typed 
   that is sent to the remote station.

\section{The Qtel Settings Dialog}
   \begin{figure}[H]
      \centering
      \includegraphics[height=80mm]{qtel-settings.pdf}
      \caption[The Qtel Settings Dialog]{Settings Dialog}
   \end{figure}

   The configuration dialog is brought up the first time the application is started.{\kern2pt}After that it can be brought up again by 
   choosing the Qtel settings menu.{\kern3pt}Most fields are self explainatory. Location is the short string that is shown in the EchoLink 
   directory server listing.{\kern3pt}The info message is the one that is sent to the remote station upon connection.{\kern2pt}In the
   Directory Server tab are some settings for the EchoLink directory server connection. Hover the mouse cursor above each field to get help 
   about it.{\kern2.5pt}That is about everything there is to say about the Qtel application. It's quite simple.

\section{Copyright}
   Copyright \textcopyright \,\,by Tobias Blomberg, SM0SVX.\\
   You are free to distribute this software under the terms of the\\
   GNU General Public License. The original ``Qtel User Documentation'' is found at\\
   \textbf{http://sourceforge.net/apps/trac/svxlink/wiki/QtelUserDocs}.\\

\end{document}                                      % end of document
